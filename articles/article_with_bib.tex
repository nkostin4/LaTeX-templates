\documentclass{article}
\usepackage[utf8]{inputenc}
\usepackage[T1]{fontenc}

% Set page geometry
\usepackage[margin=2.5cm,headsep=0.5cm]{geometry}

% AMS and mathtools
\usepackage{amsmath,amsthm,amssymb,marvosym,mathrsfs,amsfonts,amscd,mathtools}

% Hyperlinks and URLs
\usepackage{url}
\usepackage{hyperref}
\hypersetup{
    colorlinks,
    citecolor=black,
    filecolor=black,
    linkcolor=black,
    urlcolor=black
}

% Colors
\usepackage{xcolor}

% Bold math
\usepackage{bm}

% Bra Ket (Dirac) Notation
\usepackage{braket}

% Slashed characters (e.g. in Dirac equation)
\usepackage{slashed}

% Clean SI Units
\usepackage{siunitx}

% Enumerate thingies
\usepackage{enumitem}

% Cancel things out in equations
\usepackage[makeroom]{cancel}

% Graphics and figures
\usepackage{graphicx}
\usepackage{wrapfig}
\usepackage{float}

% Caption figures and tables
\usepackage{caption,subcaption}

% Generate symbols
\usepackage{textcomp} % Avoid output errors
\usepackage{gensymb}

% Make multiple rows in a table
\usepackage{multirow}

% Booktabs tables
\usepackage{booktabs}

% Useful frames
\usepackage{mdframed}

% Comment-out large sections
\usepackage{comment}

% No auto-indent
\setlength{\parindent}{0pt}

% Asymptote - 3D vector graphics
\usepackage{asymptote}

% Tikz Package Stuff
\usepackage{pgf,tikz,pgfplots}
\usepackage{tikz-3dplot}
% Use various tikz libraries
\usetikzlibrary{decorations.pathmorphing, decorations.markings, decorations.pathreplacing, patterns} % Decorate paths!
\usetikzlibrary{calc}
\usetikzlibrary{scopes}
\usetikzlibrary{angles, quotes}
\usetikzlibrary{svg.path}
\usetikzlibrary{arrows, arrows.meta}
\usetikzlibrary{fadings}
% pgfplots package settings
\pgfplotsset{compat=1.15}
% \pgfplotsset{width=10cm,compat=1.9} % From overleaf I think

% Awesome circled numbers
\newcommand*\circled[4]{\tikz[baseline=(char.base)]{\node[shape=circle, fill=#2, draw=#3, text=#4, inner sep=2pt] (char) {#1};}}

% Control size of text
\usepackage{relsize}

% Extend conditional commands
\usepackage{xifthen}

% Scale math by size
\newcommand*{\Scale}[2][4]{\scalebox{#1}{\ensuremath{#2}}}

% Big integrals
\usepackage{bigints}

% Generate blind text
\usepackage{blindtext}

% Useful symbols
\usepackage{marvosym}


%%%% THEOREMS %%%%
\newenvironment{theorem}[1][\hspace{-0.36em}]
{
    \begin{mdframed}[backgroundcolor=black!4, align=center, userdefinedwidth=40em, topline=false, bottomline = false, leftline = false, rightline = false, frametitle = {#1 Theorem}]
}
{
    \end{mdframed}
}

%%%% LEMMAS %%%%
\newenvironment{lemma}[1][\hspace{-0.36em}]
{
    \begin{mdframed}[backgroundcolor=black!4, align=center, userdefinedwidth=40em, topline=false, bottomline = false, leftline = false, rightline = false, frametitle = {#1 Lemma}]
}
{
    \end{mdframed}
}

%%%% COROLLARY %%%%
\newenvironment{corollary}[1][\hspace{-0.36em}]
{
    \begin{mdframed}[backgroundcolor=black!4, align=center, userdefinedwidth=40em, topline=false, bottomline = false, leftline = false, rightline = false, frametitle = {#1 Corollary}]
}
{
    \end{mdframed}
}

%%%% DEFINITIONS %%%%
\newenvironment{definition}[1][\hspace{-0.36em}]
{
    \begin{mdframed}[backgroundcolor=black!4, align=center, userdefinedwidth=40em, topline=false, bottomline = false, leftline = false, rightline = false, frametitle = {#1 Definition}]
}
{
    \end{mdframed}
}

%%%% PROPOSITION %%%%
\newenvironment{proposition}
{
    \begin{mdframed}[backgroundcolor=black!4, align=center, userdefinedwidth=40em, topline=false, bottomline = false, leftline = false, rightline = false, frametitle = {Proposition}]
}
{
    \end{mdframed}
}

% Change end-of-proof symbol
\renewcommand\qedsymbol{$\blacksquare$}


%%%% BLACKBOARD BOLD %%%%
\newcommand{\bbN}{\mathbb{N}} % Natural numbers
\newcommand{\bbZ}{\mathbb{Z}} % Zahlen
\newcommand{\bbQ}{\mathbb{Q}} % Rational numbers
\newcommand{\bbR}{\mathbb{R}} % Real numbers
\newcommand{\bbC}{\mathbb{C}} % Complex numbers
\DeclareSymbolFont{bbold}{U}{bbold}{m}{n} % Identity matrix
\DeclareSymbolFontAlphabet{\mathbbold}{bbold} % Identity matrix
\newcommand{\identitymatrix}{\mathbbold{1}} % Identity matrix


%%%% CODE LISTING %%%%
\usepackage{listings}
\definecolor{greencomments}{HTML}{00BA00}
\definecolor{graynumbers}{HTML}{4F4F4F}
\definecolor{purplestrings}{HTML}{AD00AA}
\definecolor{backgroundcolor}{HTML}{E8E8E8}
\lstdefinestyle{nkostin}{
    backgroundcolor=\color{backgroundcolor},
    commentstyle=\color{greencomments},
    keywordstyle=\color{blue},
    numberstyle=\tiny\color{graynumbers},
    stringstyle=\color{purplestrings},
    basicstyle=\footnotesize,
    breakatwhitespace=false,
    breaklines=true,
    captionpos=b,
    keepspaces=true,
    numbers=left,
    numbersep=5pt,
    showspaces=false,
    showstringspaces=false,
    showtabs=false,
    tabsize=2
}
\lstset{style=nkostin}

%%%% UNIT BASIS VECTORS %%%%
\newcommand{\ihat}{\bm{\hat{\imath}}} % Cartesian i hat (x-direction)
\newcommand{\jhat}{\bm{\hat{\jmath}}} % Cartesian j hat (y-direction)
\newcommand{\khat}{\bm{\hat{k}}} % Cartesian k hat (z-direction)
\newcommand{\rhat}{\bm{\hat{r}}} % Spherical r hat
\newcommand{\phihat}{\bm{\hat{\phi}}} % Spherical phi hat
\newcommand{\thetahat}{\bm{\hat{\theta}}} % Spherical theta hat
\newcommand{\nhat}{\bm{\hat{n}}} % Unit normal vector
\newcommand{\rhohat}{\bm{\hat{\rho}}} % Cylindrical rho hat
\newcommand{\zhat}{\bm{\hat{z}}} % Cylindrical z hat


%%%% COLORS: DEFINITIONS AND COMMANDS %%%%

\definecolor{nkostincolor}{HTML}{040080} % my favorite color
\newcommand{\nkostincolor}{\color{nkostincolor}}

%%%% RED %%%%
\definecolor{lightsalmon}{HTML}{FFA07A}
\definecolor{salmon}{HTML}{FA8072}
\definecolor{darksalmon}{HTML}{E9967A}
\definecolor{lightcoral}{HTML}{F08080}
\definecolor{indianred}{HTML}{CD5C5C}
\definecolor{crimson}{HTML}{DC143C}
\definecolor{firebrick}{HTML}{B22222}
\definecolor{red}{HTML}{FF0000}
\definecolor{darkred}{HTML}{8B0000}

%%%% ORANGE %%%%
\definecolor{coral}{HTML}{FF7F50}
\definecolor{tomato}{HTML}{FF6347}
\definecolor{orangered}{HTML}{FF4500}
\definecolor{gold}{HTML}{FFD700}
\definecolor{orange}{HTML}{FFA500}
\definecolor{darkorange}{HTML}{FF8C00}

%%%% YELLOW %%%%
\definecolor{lightyellow}{HTML}{FFFFE0}
\definecolor{lemonchiffon}{HTML}{FFFACD}
\definecolor{lightgoldenrodyellow}{HTML}{FAFAD2}
\definecolor{papayawhip}{HTML}{FFEFD5}
\definecolor{moccasin}{HTML}{FFE4B5}
\definecolor{peachpuff}{HTML}{FFDAB9}
\definecolor{palegoldenrod}{HTML}{EEE8AA}
\definecolor{khaki}{HTML}{F0E68C}
\definecolor{darkkhaki}{HTML}{BDB76B}
\definecolor{yellow}{HTML}{FFFF00}

%%%% GREEN %%%%
\definecolor{lawngreen}{HTML}{7CFC00}
\definecolor{chartreuse}{HTML}{7FFF00}
\definecolor{limegreen}{HTML}{32CD32}
\definecolor{lime}{HTML}{00FF00}
\definecolor{forestgreen}{HTML}{228B22}
\definecolor{green}{HTML}{008000}
\definecolor{darkgreen}{HTML}{006400}
\definecolor{greenyellow}{HTML}{ADFF2F}
\definecolor{yellowgreen}{HTML}{9ACD32}
\definecolor{springgreen}{HTML}{00FF7F}
\definecolor{mediumspringgreen}{HTML}{00FA9A}
\definecolor{lightgreen}{HTML}{90EE90}
\definecolor{palegreen}{HTML}{98FB98}
\definecolor{darkseagreen}{HTML}{8FBC8F}
\definecolor{mediumseagreen}{HTML}{3CB371}
\definecolor{seagreen}{HTML}{2E8B57}
\definecolor{olive}{HTML}{808000}
\definecolor{darkolivegreen}{HTML}{556B2F}
\definecolor{olivedrab}{HTML}{6B8E23}

%%%% CYAN %%%%
\definecolor{lightcyan}{HTML}{E0FFFF}
\definecolor{cyan}{HTML}{00FFFF}
\definecolor{aqua}{HTML}{00FFFF}
\definecolor{aquamarine}{HTML}{7FFFD4}
\definecolor{mediumaquamarine}{HTML}{66CDAA}
\definecolor{paleturquoise}{HTML}{AFEEEE}
\definecolor{turquoise}{HTML}{40E0D0}
\definecolor{mediumturquoise}{HTML}{48D1CC}
\definecolor{darkturquoise}{HTML}{00CED1}
\definecolor{lightseagreen}{HTML}{20B2AA}
\definecolor{cadetblue}{HTML}{5F9EA0}
\definecolor{darkcyan}{HTML}{008B8B}
\definecolor{teal}{HTML}{008080}

%%%% BLUE %%%%
\definecolor{powderblue}{HTML}{B0E0E6}
\definecolor{lightblue}{HTML}{ADD8E6}
\definecolor{lightskyblue}{HTML}{87CEFA}
\definecolor{skyblue}{HTML}{87CEEB}
\definecolor{deepskyblue}{HTML}{00BFFF}
\definecolor{lightsteelblue}{HTML}{B0C4DE}
\definecolor{dodgerblue}{HTML}{1E90FF}
\definecolor{cornflowerblue}{HTML}{6495ED}
\definecolor{steelblue}{HTML}{4682B4}
\definecolor{royalblue}{HTML}{4169E1}
\definecolor{blue}{HTML}{0000FF}
\definecolor{mediumblue}{HTML}{0000CD}
\definecolor{darkblue}{HTML}{00008B}
\definecolor{navy}{HTML}{000080}
\definecolor{midnightblue}{HTML}{191970}
\definecolor{mediumslateblue}{HTML}{7B68EE}
\definecolor{slateblue}{HTML}{6A5ACD}
\definecolor{darkslateblue}{HTML}{483D8B}

%%%% PURPLE %%%%
\definecolor{lavender}{HTML}{E6E6FA}
\definecolor{thistle}{HTML}{D8BFD8}
\definecolor{plum}{HTML}{DDA0DD}
\definecolor{violet}{HTML}{EE82EE}
\definecolor{orchid}{HTML}{DA70D6}
\definecolor{fuchsia}{HTML}{FF00FF}
\definecolor{magenta}{HTML}{FF00FF}
\definecolor{mediumorchid}{HTML}{BA55D3}
\definecolor{mediumpurple}{HTML}{9370DB}
\definecolor{blueviolet}{HTML}{8A2BE2}
\definecolor{darkviolet}{HTML}{9400D3}
\definecolor{darkorchid}{HTML}{9932CC}
\definecolor{darkmagenta}{HTML}{8B008B}
\definecolor{purple}{HTML}{800080}
\definecolor{indigo}{HTML}{4B0082}

%%%% PINK %%%%
\definecolor{pink}{HTML}{FFC0CB}
\definecolor{lightpink}{HTML}{FFB6C1}
\definecolor{hotpink}{HTML}{FF69B4}
\definecolor{deeppink}{HTML}{FF1493}
\definecolor{palevioletred}{HTML}{DB7093}
\definecolor{mediumvioletred}{HTML}{C71585}

%%%% WHITE %%%%
\definecolor{white}{HTML}{FFFFFF}
\definecolor{snow}{HTML}{FFFAFA}
\definecolor{honeydew}{HTML}{F0FFF0}
\definecolor{mintcream}{HTML}{F5FFFA}
\definecolor{azure}{HTML}{F0FFFF}
\definecolor{aliceblue}{HTML}{F0F8FF}
\definecolor{ghostwhite}{HTML}{F8F8FF}
\definecolor{whitesmoke}{HTML}{F5F5F5}
\definecolor{seashell}{HTML}{FFF5EE}
\definecolor{beige}{HTML}{F5F5DC}
\definecolor{oldlace}{HTML}{FDF5E6}
\definecolor{floralwhite}{HTML}{FFFAF0}
\definecolor{ivory}{HTML}{FFFFF0}
\definecolor{antiquewhite}{HTML}{FAEBD7}
\definecolor{linen}{HTML}{FAF0E6}
\definecolor{lavenderblush}{HTML}{FFF0F5}
\definecolor{mistyrose}{HTML}{FFE4E1}

%%%% GRAY %%%%
\definecolor{gainsboro}{HTML}{DCDCDC}
\definecolor{lightgray}{HTML}{D3D3D3}
\definecolor{silver}{HTML}{C0C0C0}
\definecolor{darkgray}{HTML}{A9A9A9}
\definecolor{gray}{HTML}{808080}
\definecolor{dimgray}{HTML}{696969}
\definecolor{lightslategray}{HTML}{778899}
\definecolor{slategray}{HTML}{708090}
\definecolor{darkslategray}{HTML}{2F4F4F}
\definecolor{black}{HTML}{000000}

%%%% BROWN %%%%
\definecolor{cornsilk}{HTML}{FFF8DC}
\definecolor{blanchedalmond}{HTML}{FFEBCD}
\definecolor{bisque}{HTML}{FFE4C4}
\definecolor{navajowhite}{HTML}{FFDEAD}
\definecolor{wheat}{HTML}{F5DEB3}
\definecolor{burlywood}{HTML}{DEB887}
\definecolor{tan}{HTML}{D2B48C}
\definecolor{rosybrown}{HTML}{BC8F8F}
\definecolor{sandybrown}{HTML}{F4A460}
\definecolor{goldenrod}{HTML}{DAA520}
\definecolor{peru}{HTML}{CD853F}
\definecolor{chocolate}{HTML}{D2691E}
\definecolor{saddlebrown}{HTML}{8B4513}
\definecolor{sienna}{HTML}{A0522D}
\definecolor{brown}{HTML}{A52A2A}
\definecolor{maroon}{HTML}{800000}

%%%% CAT %%%%
\newcommand{\cat}[1][]{\tikz \fill [scale=1ex/500,yscale=1,#1] svg "M6125 12741 c-387 -139 -597 -254 -908 -495 -233 -181 -331 -236 -422 -236 -17 0 -69 11 -115 24 -115 33 -387 82 -540 97 -236 22 -573 0 -849 -56 -161 -33 -227 -39 -285 -25 -32 7 -108 48 -220 117 -381 238 -783 418 -1165 522 l-74 20 7 -87 c47 -605 193 -1137 396 -1443 l40 -60 -25 -65 c-38 -101 -93 -307 -116 -439 -18 -97 -22 -161 -23 -330 0 -213 11 -317 49 -441 26 -86 51 -79 -277 -76 -331 4 -697 -10 -1025 -38 -215 -19 -564 -59 -572 -66 -2 -2 -1 -9 2 -17 4 -11 27 -10 138 4 476 63 1214 101 1600 84 l167 -7 36 -87 c20 -47 55 -121 80 -163 l43 -77 -36 -5 c-20 -3 -92 -13 -161 -21 -513 -65 -1082 -183 -1506 -315 -194 -59 -234 -76 -234 -95 0 -8 1 -15 3 -15 2 0 57 18 122 40 266 90 566 168 900 234 276 55 413 76 891 141 l41 6 58 -77 c31 -42 88 -109 126 -149 l69 -72 -118 -43 c-265 -97 -648 -277 -882 -415 -179 -105 -370 -235 -370 -251 0 -8 4 -14 9 -14 4 0 69 40 142 89 339 225 742 426 1156 577 l93 33 93 -94 c214 -215 287 -380 287 -652 0 -260 -53 -544 -204 -1093 -70 -254 -124 -480 -156 -652 -157 -835 -108 -1553 159 -2358 89 -266 155 -431 312 -782 274 -611 310 -727 354 -1146 22 -205 36 -683 33 -1072 l-3 -345 -88 -7 c-109 -8 -232 -32 -290 -57 -66 -28 -111 -73 -143 -143 -26 -56 -29 -74 -29 -158 0 -74 4 -103 19 -129 67 -123 257 -196 611 -235 1310 -143 2603 -163 3865 -61 684 56 1558 169 1680 218 349 141 670 737 925 1721 185 712 354 1713 371 2201 24 650 -166 1275 -569 1880 -207 310 -356 484 -807 940 -477 483 -631 662 -805 935 -143 224 -273 537 -311 747 -35 198 -28 460 18 673 47 221 145 445 262 601 203 269 554 486 916 565 66 15 125 19 260 18 157 0 185 -3 274 -27 228 -61 359 -160 386 -292 26 -124 -55 -283 -190 -373 -135 -91 -278 -109 -464 -59 -131 35 -183 41 -236 27 -75 -19 -115 -52 -150 -124 -30 -61 -32 -71 -28 -144 6 -97 36 -158 119 -238 71 -67 209 -135 321 -158 99 -20 253 -20 358 0 267 50 581 244 737 454 209 281 260 673 133 1013 -70 186 -311 431 -525 533 -206 98 -593 140 -925 99 -549 -68 -1041 -298 -1400 -654 -243 -242 -405 -492 -496 -769 l-37 -113 -56 6 c-135 13 -486 25 -753 25 -271 0 -289 1 -284 18 44 150 60 274 60 457 0 244 -33 410 -129 650 l-44 109 49 66 c252 342 406 715 456 1103 22 176 12 628 -15 627 -3 -1 -78 -27 -166 -59z m709 -3021 c88 -6 161 -11 162 -13 1 -1 -6 -42 -17 -92 -25 -119 -49 -305 -49 -377 0 -32 -3 -58 -6 -58 -3 0 -63 13 -132 29 -251 58 -593 119 -872 156 -69 9 -135 18 -147 21 -21 4 -20 8 32 117 29 61 61 138 71 169 18 53 21 57 54 62 57 7 732 -3 904 -14z m-604 -436 c173 -28 390 -71 666 -131 20 -5 21 -13 28 -166 27 -621 177 -1043 583 -1650 253 -378 580 -768 1023 -1222 273 -279 343 -357 438 -484 238 -315 387 -652 453 -1019 72 -406 36 -1045 -97 -1712 -160 -804 -476 -1596 -709 -1772 -91 -69 -196 -83 -285 -38 -126 65 -154 220 -111 605 12 105 37 318 56 475 62 523 75 690 75 955 0 641 -136 1169 -474 1844 -344 687 -592 1023 -1179 1597 -293 287 -426 407 -721 653 -143 120 -297 251 -340 292 -240 224 -359 512 -373 899 -4 114 -2 159 11 210 21 86 73 186 148 288 54 75 64 83 87 79 46 -10 341 -136 516 -222 207 -101 374 -196 564 -322 107 -71 146 -92 154 -84 8 8 8 14 2 19 -364 252 -809 488 -1172 621 -40 15 -60 27 -56 35 4 6 41 61 84 121 42 61 88 131 103 156 l26 46 143 -19 c78 -10 239 -34 357 -54Z";}


% Bibliography Management
\usepackage[backend=biber,sorting=none]{biblatex}
\addbibresource{ref.bib}
\BiblatexSplitbibDefernumbersWarningOff

\begin{document}

Nicholas Kostin \\
\url{https://github.com/nkostin4}

% \tableofcontents

\section{Introduction}

This brief document is meant to show the best paradigm for bibliography management with {\LaTeX}. This guide contains examples that touch on perturbation theory, the Aharonov-Bohm effect, and electromagnetic radiation --- all topics in the field of theoretical physics.

\vspace{1em}

On an entirely unrelated note, it is worth remembering that ``one of the biggest issues of modern post-war institutionalized science is that the funding and peer-review mechanism is self-reinforcing \ldots [which] creates a community of "scientists" who are more and more incestuous and generally oblivious not just to other possibilities of inquiry, but don't even have to be aware of their own priors or assumptions.'' \cite{fragile}. 

\section{Perturbation Theory --- Nondegenerate Case}

Perturbation theory is a useful technique to obtain approximate solutions to eigenvalue-eigenvector problems that are too complicated to be solved exactly \cite{byronfuller}. Suppose we want to find the eigenvalues and eigenvectors of a self-adjoint operator $A$ which may be split up into the sum of two self-adjoint operators:
\begin{equation*}
    A = A_o + \epsilon A_1.
\end{equation*}

We assume that we know something about $A_o$. For example, we might know some eigenvalue and its associated eigenvector. We would like to find the corresponding eigenvalue and eigenvector of $A$, given that the influence of $A_o$ is known to be predominant. We have already anticipated this situation be writing the ``perturbing'' operator in the form of $\epsilon A_1$, where $\epsilon$ is some small parameter \cite{byronfuller}.

\section{The Aharonov-Bohm Effect}

Classically, a particle can't sense potentials, only fields. In other words, the actual value of a potential --- whether it be $\bm{A}$ or $\phi$ --- isn't supposed to matter. In quantum mechanics, we want to obtain the eigenfunctions describing allowed particle states by solving $H \psi = E \psi$, where the Hamiltonian reads
\begin{equation*}
    H = \frac{1}{2m} \left( \frac{\hbar}{i} \nabla - q \bm{A} \right)^2 + q\phi.
\end{equation*}

Observe that the Hamiltonian is expressed in terms of potentials, not fields. It should then stand to reason that a particle whose behavior is governed quantum mechanically can ``sniff'' nearby magnetic fields via the resulting vector potential $\bm{A}$.

\subsection{General Description}

Suppose $\psi'$ is an eigenstate of
\begin{equation*}
	H = \frac{\hbar^2}{2m} \nabla^2 + q\psi.
\end{equation*}

Observe that this Hamiltonian doesn't have a vector potential present. Say $\psi'$ is a basis state. Now suppose we consider a new region where $\bm{B}$ is zero (so the vector potential $\bm{A}$ is still curl-free), but now there is a non-zero vector potential:
\begin{equation*}
	\nabla \times \bm{A} = \bm{0}, \qquad \bm{A} \neq \bm{0}.
\end{equation*}

In this regime, the eigenstate, say, $\psi$, of this new Hamiltonian can be built from the old $\psi'$ eigenstates. In fact, these differ only by a phase factor:
\begin{equation*}
	\psi = e^{ig} \psi', \qquad \text{where } g \equiv \frac{q}{\hbar} \int \bm{A} \cdot d\bm{\ell}.
\end{equation*}

All of this is pretty abstract, so let's summarize what we have so far: the wavefunctions for particles in a vector potential --- in a region where there is zero $\bm{B}$-field (and maybe even zero $\bm{E}$-field) --- pick up phases when they travel through that region. Different phases depending on the paths involved. But no problem, right? The square magnitude of the new $\psi$ makes the new phase go away, right? Unless, we have more than one particle. Then we get observable interference that depends on the value of the vector potential.

\subsection{Aharonov and Bohm's Experiment}

Aharonov and Bohm proposed an experiment in which a stream of electrons is split in two, and passes either side of a solenoid \cite{aharonovbohm}. A long solenoid has a uniform magnetic field within it, and nearly zero magnetic field outside. However, the vector potential $\bm{A}$ at a distance $r$ from the center of the solenoid is not zero. In particular,
\begin{equation*}
	\bm{A} = \frac{\pi a^2 B}{2\pi r} \phihat,
\end{equation*}

where $a$ is the radius of the solenoid and $B$ is the magnitude of the magnetic field. One of the electron beams in the experiment travels along this vector potential, while the other moves against it. As such, there is a phase accumulation:
\begin{equation*}
    g = \frac{q}{\hbar} \int \bm{A} \cdot d\bm{\ell} = \frac{q\ \pi a^2\ B}{2\pi\hbar} \int \left( \frac{1}{r}\ \phihat \right) \cdot \left( r\ \phihat\ d\phi \right) = \pm \frac{q\ \pi a^2\ B}{2\hbar}.
\end{equation*}

Say, the electrons traveling along the vector potential accumulate a positive phase, while the electrons traveling against the vector potential accumulate an equal and opposite negative phase, such that the phase difference is $\displaystyle \frac{q\ \pi a^2\ B}{\hbar}$. Herein lies the conclusion that Yakir Aharonov and David Bohm reached --- for a charged quantum particle moving though space, it is insufficient to know the local electromagentic field to predict the time evolution of the particle's wavefunction. As cool as this is, the Aharonov Bohm effect is nothing more than a special case of the broader geometric phase \cite{aharonovbohm}.

\section{Radiation}

When charges accelerate, their fields can transport energy irreversibly out to infinity --- a process we call radiation \cite{griffiths}. Let's assume the source is localized near the orgin. Our objective is to calculate the energy it is radiating at time $t_o$. Imagine a gigantic sphere, out at radius $r$. The power passing through its surface is the integral of the Poynting vector:
\begin{equation*}
    P(r, t) = \oint \bm{S} \cdot d\bm{a} = \frac{1}{\mu_o} \oint \left( \bm{E} \times \bm{B} \right) \cdot d\bm{a}.
\end{equation*}

Because electromagnetic ``news'' travels at the speed of light, this energy actually left the source at the earlier time $t_o = t - r/c$, so the power radiated is
\begin{equation*}
    P_{\text{rad}} \left( t_o \right) = \lim_{r \to \infty} P \left( r, t_o + \frac{r}{c} \right)
\end{equation*}

(with $t_o$ held constant). This is the energy (per unit time) that is carried away and never comes back \cite{griffiths}.

% Print entire bibliography - title it 'References'
\printbibliography[heading=bibintoc,title={References}]

% Only books bibliography
\printbibliography[heading=subbibintoc,type=book,title={Books only}]

% Only Physics Resources
\printbibliography[heading=subbibintoc,keyword={physics},title={Physics-related only}]

\end{document}
