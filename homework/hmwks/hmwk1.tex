\documentclass[class=report,crop=false]{standalone}
\usepackage[margin=2.5cm,headsep=0.5cm]{geometry}
\input{../preamble}

\begin{document}

\section*{A Vector in a Hilbert space}
\addtocounter{section}{1}
\addcontentsline{toc}{section}{A Vector in a Hilbert space}
\setcounter{equation}{0}

The state of a quantum system can be described by a vector in a Hilbert space. Once a basis is chosen for the Hilbert space, the quantum state can be represented by a column vector with complex number entries. Using a column vector to describe a quantum state has the drawbrack of being basis-dependent. Suppose the basis of a Hilbert space is formed by $\displaystyle \left\{ \chi_1, \chi_2, \cdots, \chi_N \right\}$. We now have two ways to represent a quantum state.

\begin{equation}
    \ket{\psi} = \sum\limits_{i = 1}^{N} a_i \ket{\chi_i} \qquad \Longleftrightarrow \qquad \bm{\psi} = \begin{pmatrix} a_1 \\ \vdots \\ a_N \end{pmatrix}
\end{equation}

\section*{The Inner Product}
\addtocounter{section}{1}
\addcontentsline{toc}{section}{The Inner Product}
\setcounter{equation}{0}

For two real column vectors $\bm{u}$ and $\bm{v}$, the inner product between them is defined as $\displaystyle \bm{u}^T \bm{v} = \sum_i u_i v_i$. This is the same as the dot product between the two corresponding vectors in the Euclidean space $\vec{u} \cdot \vec{v}$. The dot product is invariant under a change of basis. For two vectors in the Hilbert space $\ket{\psi}$ and $\ket{\phi}$, we denote their inner product as $\braket{\psi | \phi}$. If a specific basis of the Hilbert space is chosen, we can represent $\ket{\psi}$ and $\ket{\phi}$ using two column vectors:
\begin{equation}
    \bm{\psi} = \begin{pmatrix} a_1 \\ \vdots \\ a_n \end{pmatrix} \qquad \text{and} \qquad \bm{\phi} = \begin{pmatrix} b_1 \\ \vdots \\ b_n \end{pmatrix}.
\end{equation}

Correspondingly, the bra $\bra{\psi}$ would be represented by the row vector
\begin{equation}
    \bm{\psi}^{\dagger} = \begin{pmatrix} a_1^* \cdots a_n^* \end{pmatrix}.
\end{equation}

The Dirac notation $\braket{\psi|\phi}$ can then be regarded as the matrix product of $\bm{\psi}^{\dagger}$ and $\bm{\phi}$:
\begin{equation}
    \braket{\psi|\phi} = \bm{\psi}^{\dagger}\bm{\phi} = \sum\limits_{i = 1}^{n} a_i^* b_i.
\end{equation}

Note that the inner product is also invariant under the change of basis.

\section*{Orthonormal Basis}
\addtocounter{section}{1}
\addcontentsline{toc}{section}{Orthonormal Basis}
\setcounter{equation}{0}

The basis vectors of a Hilbert space are typically made of states that are normalized and orthogonal to each other. For example, if $\left\{ \chi_1, \chi_2, \cdots, \chi_N \right\}$ form a basis, then for any $i, j \in \left\{ 1, 2, \cdots, N \right\}$, we have
\begin{equation}
    \braket{\chi_i | \chi_j} = \delta_{ij} = \begin{cases} \displaystyle 0, & \text{if } i \neq j \\ \displaystyle 1, & \text{if } i = j \end{cases}
\end{equation}

Such as basis is said to be \emph{orthonormal}.

\section*{Time Evolution}
\addtocounter{section}{1}
\addcontentsline{toc}{section}{Time Evolution}
\setcounter{equation}{0}

For a \colorbox{REDE}{time-independent} Hamiltonian $\hat{H}$, the state of the system at time $t$ is given by the solution to the Schr{\"o}dinger equation $\displaystyle i\hbar \frac{d \ket{\psi(t)}}{dt} = \hat{H} \ket{\psi(t)}$:

\begin{equation}
    \Ket{\psi(t)} = e^{-i\hat{H}t / \hbar} \ket{\psi(0)}.
\end{equation}

\end{document}
